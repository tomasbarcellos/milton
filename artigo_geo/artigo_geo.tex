\documentclass[article]{jss}
\usepackage[utf8]{inputenc}

\providecommand{\tightlist}{%
  \setlength{\itemsep}{0pt}\setlength{\parskip}{0pt}}

\author{
FirstName LastName\\University/Company \And Second Author\\Affiliation
}
\title{A Capitalized Title: Something about a Package \pkg{foo}}

\Plainauthor{FirstName LastName, Second Author}
\Plaintitle{A Capitalized Title: Something about a Package foo}
\Shorttitle{\pkg{foo}: A Capitalized Title}

\Abstract{
The abstract of the article.
}

\Keywords{keywords, not capitalized, \proglang{Java}}
\Plainkeywords{keywords, not capitalized, Java}

%% publication information
%% \Volume{50}
%% \Issue{9}
%% \Month{June}
%% \Year{2012}
%% \Submitdate{}
%% \Acceptdate{2012-06-04}

\Address{
    FirstName LastName\\
  University/Company\\
  First line Second line\\
  E-mail: \email{name@company.com}\\
  URL: \url{http://rstudio.com}\\~\\
    }

% Pandoc header

\usepackage{amsmath}

\begin{document}

\hypertarget{introduction}{%
\section{Introduction}\label{introduction}}

SHORT INTRO {[}Geo data is increasingly important, review{]}\\
Our aim was to develop a R package to help researcher (in the clinical,
social sciences, etc) in dealing with geospatial data.\\
We searched for geospatial packages in the CRAN repository. A short
description of each package is avaliable in Table\ref{table1}. Few of
them are intended for the researcher who is not familiar with more
complex coding.\\
Since tabular is the most used type of dataset, our package offers
functions to extract and transform data from shapefiles to regular
datasets.\\
@ Merge tabular and geospatial data\\
We also included functions to calculate distances between points in a
shapefile. Our package calculate distances from raw addresses or
coordinates (latidude, longitude). @ Minimum Distance\\
@ Given a list of address, returns the closest one

\begin{table}[h]
\begin{tabular}{|l|l|}
\hline
\textbf{Package} & \textbf{Description} \\ \hline
APfun & Utilities for handling shapefiles and polygons (merge, etc.) \\ \hline
aspace & \vbox{\hbox{\strut Centrographic statistics, computational geometry, and home}\hbox{\strut range ecology for exploring human activities in cities}} \\ \hline
bbo & Geographical distribution of biological organisms \\ \hline
biogeo & Conversion of standards and units, outliers and error detection \\ \hline
BPEC & DNA Analysis of clusters and migration patterns using mitochondrial DNA \\ \hline
btb & Kernel density estimator for urban environment with geolocal data \\ \hline
ClustGeo & Hierarchical clustering with Ward criterion for geolocal data \\ \hline
gcKrig & Geospatial analysis with gaussian copulas \\ \hline
gdalUtils & Utilities for gdal (Geospatial Data Abstraction Library)  \\ \hline
gdistance & Routes and distances in heterogeneous spaces using grids \\ \hline
gear & Statistical methods for spatial analysis (e.g. glm) \\ \hline
geoaxe & Cuts 'geospatial' objects into disjoint areas \\ \hline
geoBayes & Bayesian analysis with geographical data \\ \hline
GeoBoxplot & Boxplot for geospatial plots  \\ \hline
geoCount & Generalized spatial models for count data \\ \hline
geodist & High performance geodesical distances  \\ \hline
geofacet & Creates grid of plots in ggplot following the contour of a map  \\ \hline
geofd & Kriging methods for predictiong of spatially dependet curves. \\ \hline
geoGAM & Spatial prediction using complex models  \\ \hline
geomerge & Merges rasters, polygons and points \\ \hline
geoR & Complete framework for geospatial analysis \\ \hline
geoRglm & geoR extension for generalized linear models  \\ \hline
georob & Linear models with spatially correlated errors and cross-validation  \\ \hline
georob & Functions for dealing with linear models with spatially correlated errors  \\ \hline
geospacom & Calculate distance matrix from shapefiles \\ \hline
geospt & Network measures, variogram  \\ \hline
geosptdb & Geostatistical analysis of spatial data \\ \hline
geostatsp & Geostatistical Analysis, satial sampling networks \\ \hline
geotools & Functions for dealing with postal codes; distances between two coordinates \\ \hline
geotoolsR & Bootstraping methods for geostatistics \\ \hline
geozoning & Zoning \\ \hline
Gmedian & Estimation of the geometric median, k-Gmedian clustering, PCA \\ \hline
gstat & Spatial and Spatio-Temporal geostatistical modelling, prediction and simulation \\ \hline
GWLelast & Geographically weighted logistic elastic net regression \\ \hline
gwrr & Geographically weighted regression \\ \hline
krige & Kriging models for geographical point-referenced data \\ \hline
mapStats & Calculation and visualization of survey data \\ \hline
nngeo & K-nearest neighboor (Clustering) for spatial data \\ \hline
pointdensityP & Point density for geospatial data \\ \hline
ramps & Bayesian geostatistical modeling of Gaussian processes \\ \hline
revgeo & Reverse geocode with APIs \\ \hline
rgdal & Bindings for the 'Geospatial' Data Abstraction Library (GDAL) \\ \hline
sgeostat & Functions for variogram estimation, variogram fitting, kriging, plotting \\ \hline
spcosa & Spatial coverage sampling and random sampling, clustering \\ \hline
vapour & Binding for GDAL \\ \hline
georob & Methods for fitting linear models with spatially correlated errors \\ \hline
geospacom DW & Distance matrix from shape files \\ \hline
geoSpectral DW & Functions for dealing with geo-spectral data \\ \hline
\label{table1}
\end{tabular}
\end{table}

@ Adicionar quais aceitam endereços como entrada

\hypertarget{an-example-with-simulated-data}{%
\section{An Example with Simulated
Data}\label{an-example-with-simulated-data}}

@ Distance to the school and school performance (school p
\textasciitilde{} distance); @ Minimum distance to two schools.



\end{document}

